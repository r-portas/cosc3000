\documentclass[]{article}

\usepackage[margin=1in]{geometry}
\usepackage{url}
\usepackage{parskip}

\begin{document}

\title{Computer Visualization Report}
\author{Roy Portas}
\date{\today}
\maketitle

\section{Introduction}

Criminology is the study of causes, nature and prevention of criminal behaviour\cite{roufa_early_nodate}.
This study is important as it allows sociologists and psychologists to not only study the when and how crime occurs, but offers a way to predict crime trends in the future.

Crime is an evident part of our society and is an active topic during elections\cite{remeikis_queensland_2016}, 
even companies like IBM\cite{noauthor_ibm_2012} and Palantir\cite{palantir_technologies_law_nodate} 
are creating products to model, predict and prevent crimes before they happen.

As technology becomes more powerful, the cost of implementing systems capable of predicting crimes becomes more feasible.
However these systems are difficult to build, as it is almost impossible to predict the future.
Nevertheless companies have started building such systems.

Crime prediction systems generally consist of the following components:
\begin{itemize}
    \item Identify areas with frequent crime\cite{noauthor_ibm_2012}
    \item Match trends in national and regional crimes with local crimes
    \item Identify circumstances for the cause of crimes\cite{noauthor_ibm_2012}
\end{itemize}

\section{Aims}

This report aims to analyse and explore crime data for Queensland. A variety of techniques and datasets will be used to explore this problem, including:

\begin{itemize}
    \item Analysis of geolocation data for crime locations and police districts
\end{itemize}

\section{Methods}

\section{Results}

\section{Limitations}

\section{Conclusions}

\section{References}

\bibliographystyle{abbrvurl}
\bibliography{bibliography}

\end{document}
