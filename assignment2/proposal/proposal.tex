\documentclass[]{article}

\usepackage[margin=1.2in]{geometry} 
\usepackage{url} 
\usepackage{parskip}
\usepackage{graphicx}
\usepackage{float}

\begin{document}

\title{Computer Graphics Project Proposal}
\author{Roy Portas - s4356084}
\date{\today}
\maketitle 

\section*{Project Goal}

To visualize global crime data on a 3d globe of the world.
The application will be interactive, allowing the user to rotate the global and zoom in, similar to how Google Earth works.

\section*{Purpose of Project}

The program will allow users to visualize crime data around the globe.
It aims to fulfil the same role as a infographic, allowing users to easily visualize and understand the data.

\section*{Computer Graphics Techniques}

The visualization will be created in WebGL 1.0, which allows rendering 3D graphics in any webpage.
WebGL is implemented in most browsers, and programs can be written using standard JavaScript.
The WebGL 1.0 API is based on OpenGL ES 2.0, so code will be written in a similar way as the course sample code.

The project will use the following computer graphic techniques:
\begin{itemize}
    \item A virtual camera which renders a viewport of the 3d scene using rasterization
    \item Displaying a 3d globe in the scene and texturing it with a texture of Earth
    \item Implement controls to rotate the globe
    \item Use point lighting behind the camera to show the globe and use specular highlighting to improve aesthetics
    \item Display the locations of crimes on the globe by translating latitude and longitudes to points in the world space
\end{itemize}

\end{document}
